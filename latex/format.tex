\chapter{Sparse matrix formats.}

\section{Eigen's compressed (column) format.}

Eigen stores sparse matrices in a compressed row or compressed column
format.  Compressed column is the default and the following discussion
focuses on the compressed column format.  The format is composed of
four vectors, the following description is paraphrased from the Eigen
tutorial (http://eigen.tuxfamily.org/dox/group__TutorialSparse.html) and
follows the nomenclature from the tutorial.  In the compressed column
format the outer dimension indices are the columns and inner indices are
the rows.  In the compressed row format these are reversed.

\begin{description}
\item[Values, $w$] the values vector stores the non-zero entries from the sparse matrix, and
typically contains some empty space to allow values to be efficiently
inserted in the future.  We refer to the length of this vector as $p$.
\item[InnerIndices, $v$] the inner indices vector is the same length as the
values vector and indicates which matrix row the value is in. 
\item[OuterStarts, $u$] the outer starts vector is equal in length to the
number of columns of $X$ + 1.  The entry $u_j$ indicates where
in $w$ the non-zero values (NZVs) for column $j$ begin.  The final entry
indexes one-past the end of $w$ and $v$.
\item[InnerNNZs, $z$] the number of entries in the inner number of non-zero values (NNZs) vector
is equal to the number of columns of $X$.  Entry $z_j$ stores the count
of non-zero values in column $X_{,j}$.
\end{description}

Since $w$ is padded to allow for fast inserts, it's always true that

\begin{equation}
p \geq NNZ
\end{equation}

For a sparse matrix the number of entries stored can be much smaller
than a dense matrix:

\begin{equation}
2p + 2n + 1 \approx 2NNZ + 2n
\end{equation}

Iterating over this matrix requires $n +\sum z$ steps---at least in the
way I can formulate, I have not looked up how Eigen does it.

\section{R's compressed (column) format.}

What R does is what R is, so we go by example:

\begin{verbatim}
>      (AA <- sparseMatrix(c(1,3:8), c(2,9,6:10), x = 7 * (1:7), dims = c(10,20)))
10 x 20 sparse Matrix of class "dgCMatrix"
                                                  
 [1,] . 7 . . .  .  .  .  .  . . . . . . . . . . .
 [2,] . . . . .  .  .  .  .  . . . . . . . . . . .
 [3,] . . . . .  .  .  . 14  . . . . . . . . . . .
 [4,] . . . . . 21  .  .  .  . . . . . . . . . . .
 [5,] . . . . .  . 28  .  .  . . . . . . . . . . .
 [6,] . . . . .  .  . 35  .  . . . . . . . . . . .
 [7,] . . . . .  .  .  . 42  . . . . . . . . . . .
 [8,] . . . . .  .  .  .  . 49 . . . . . . . . . .
 [9,] . . . . .  .  .  .  .  . . . . . . . . . . .
[10,] . . . . .  .  .  .  .  . . . . . . . . . . .
> str(AA)
Formal class 'dgCMatrix' [package "Matrix"] with 6 slots
  ..@ i       : int [1:7] 0 3 4 5 2 6 7
  ..@ p       : int [1:21] 0 0 1 1 1 1 2 3 4 6 ...
  ..@ Dim     : int [1:2] 10 20
  ..@ Dimnames:List of 2
  .. ..$ : NULL
  .. ..$ : NULL
  ..@ x       : num [1:7] 7 21 28 35 14 42 49
  ..@ factors : list()
> diff(AA@p)
 [1] 0 1 0 0 0 1 1 1 2 1 0 0 0 0 0 0 0 0 0 0

\end{verbatim}

In R's format the slot $x$ matches the Eigen values vector, the slot $i$ matches 
our inner indices vector (down to the zero-based indexing), the slot $p$ indicates 
the index of $X_{i,j}$ in the slot $x$, which matches Eigen's outer starts vector,
but the number of non-zero values per column must be calculated as \verb?dim(X@p)?.
We implement a short R function to take an R sparse matrix and return a list matching the
Eigen format:

\begin{verbatim}

\end{verbatim}

\section{Stan-language compressed column format.}

The current Stan language does not contain a sparse format so we instead
store each sparse matrix as the four vectors of Eigen's compressed
column (CC) format.  The data members required to delcare the sparse
matrix $X$ would be as follows:

\wrappingon
\begin{verbatim}
data {
	int m;
	int n;
	int p;
	vector[p] w;
	vector[p] v;
	vector[n+1] u;
	vector[n] z
}
\end{verbatim}
\wrappingoff

Sparse matrix operations using this format are defined in the following
sections.





